%% LyX 2.3.6.1 created this file.  For more info, see http://www.lyx.org/.
%% Do not edit unless you really know what you are doing.
\documentclass[english]{article}
\usepackage[T1]{fontenc}
\usepackage[latin9]{inputenc}
\usepackage[a4paper]{geometry}
\geometry{verbose,tmargin=2cm,bmargin=2cm,lmargin=2cm,rmargin=2cm}
\setlength{\parskip}{\smallskipamount}
\setlength{\parindent}{0pt}
\usepackage{amssymb}
\usepackage{babel}
\begin{document}

\section{EKF in a nutshell}

Extended Kalman Filter (EKF) is an modification of the well known
Kalman Filter for the case of nonlinear dynamics. The Kalman Filter
itself is an optimal state estimation algorithm for linear dynamics
and measurement models which experience zero-mean Gaussian noise.
In the EKF context however, neither the dynamical model nor the measurement
model have to be linear, nonetheless, linearized dynamics and measurements
models are still required within the filter. One dynamical system
can be described as follows:

\begin{equation}
\dot{{\bf x}}=f(t,{\bf x})+{\bf w}
\end{equation}

in which the function $f:\mathbb{R}\times\mathbb{R}^{n}\rightarrow\mathbb{R}^{n}$
and ${\bf w}$ is a vector of $n$ normally distributed random variables
with zero mean and a covariance matrix ${\bf Q}$. The same system
can be written in discrete time as:

\begin{equation}
{\bf x}_{k+1}={\bf F}(t_{k},t_{k+1},{\bf x}){\bf x}+{\bf w}_{k}
\end{equation}

where ${\bf F}:\mathbb{R}\times\mathbb{R}\times\mathbb{R}^{n}\rightarrow\mathbb{R}^{n\times n}$is
the state transition matrix. A first order approximation of the state
transition matrix is found below:

\begin{equation}
{\bf F}(t_{k},t_{k+1},{\bf x})={\bf I}+\frac{\partial f(t,{\bf x})}{\partial t}\cdot(t_{k}-t_{k+1})\label{eq:STM}
\end{equation}

The measurements are modeled as follows:

\begin{equation}
{\bf z}=h(t,{\bf x})+{\bf v}
\end{equation}

in which the function $h:\mathbb{R}\times\mathbb{R}^{n}\rightarrow\mathbb{R}^{m}$
and ${\bf v}$ is a vector of $m$ normally distributed random variables
with zero mean and a covariance matrix ${\bf R}$. One other important
matrix that needs to be available for the EKF algorithm is the $H(t,{\bf x})=\frac{\partial h(t,{\bf x})}{\partial t}$.

\section{Example details}

In the context of this example, the position and velocity of a vehicle
moving along a line are to be estimated. The linear acceleration of
the system is modeled follows:

\begin{equation}
\ddot{x}=-\sin(t)\label{eq:dynamics}
\end{equation}

letting $x_{1}=x$ and $x_{2}=\dot{x}$, the dynamical model can be
written as:

\begin{equation}
\left[\begin{array}{c}
\dot{x_{1}}\\
\dot{x_{2}}
\end{array}\right]=\left[\begin{array}{c}
x_{2}\\
-\sin(t)
\end{array}\right]
\end{equation}

The state transition matrix of the system is written according to
the approximation in (\ref{eq:STM}) as:

\begin{equation}
{\bf F}(t_{k},t_{k+1},{\bf x})=\left[\begin{array}{cc}
1 & t_{k}-t_{k+1}\\
0 & 1
\end{array}\right]
\end{equation}

The system is assumed to measure only the position of the vehicle,
hence:

\[
{\bf z}=x_{1}\quad\Rightarrow\quad H(t,{\bf x})=\left[\begin{array}{cc}
1 & 0\end{array}\right]
\]

The ground truth of the system is synthesized by analytically solving
(\ref{eq:dynamics}) and adding zero-mean Gaussian noise to the velocity
(which in-turn affects the position), while the measurements where
synthesized by adding zero-mean Gaussian noise to the position ground
truth.
\end{document}
